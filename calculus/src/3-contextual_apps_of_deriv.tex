\documentclass[11pt]{article}
\usepackage[margin=1.5in]{geometry}
\usepackage{graphicx}
\usepackage{float}
\usepackage{parskip}
\usepackage{amsmath}
\usepackage{pgfplots}
\pgfplotsset{width=10cm, compat=1.9}

\begin{document}

\textbf{\Huge Contextual Applications of \\ Derivatives}

Athan Zhang \& Jeffrey Chen

\section{Basic Rate of Change Applications}
One of the most basic contextual applications of derivatives lies in modeling the rate of change of functions. In these problems, you will be given a function that models a contextual quantity and will be asked to investigate the rate of change of the function. Usually, these problems are straightforward and only require you to take one or two derivatives.

% add volume or melting example

\subsection{Rectilinear Motion}
Rectilinear motion refers to the motion of a particle along a straight line. It can be analyzed using the derivative. Relevant concepts include position, displacement, distance, velocity, and acceleration. Students who have dabbled in physics may already understand the relationship between these values. They are defined below:

\begin{enumerate}
    \item \textbf{Position:} The position of a particle refers to its location on a straight line, and can be represented as a function of time elapsed $t$, i.e. $s(t)$.
    \item \textbf{Displacement:} The displacement of a particle refers to the change in position of a particle. It includes direction, meaning it can be negative. It can be represented as a function of time elapsed $t$, i.e. $d(t)$.
    \item \textbf{Distance:} The distance covered by a particle refers to the absolute value of a particle's displacement or change in position, hence it can only be positive. 
    \item \textbf{Velocity:} The velocity of a particle is the rate at which its position changes with respect to time. It can be represented as the derivative of position or displacement (the derivatives of these two quantities are equivalent), i.e. $v(t)=s'(t)=d'(t)$.
    \item \textbf{Acceleration:} The acceleration of a particle is the rate at which its velocity changes with respect to time. It can be represented as the derivative of velocity or the second derivative of position or displacement, i.e. $a(t)=v'(t)=s''(t)=d''(t)$.
\end{enumerate}

Rectilinear motion can be analyzed with the use of derivatives to model changes over time. It is a key application of derivatives in the real world.

% add motion example

\section{Implicit Differentiation}

\subsection{Implicit Functions}
A function $y$ of $x$ is said to be defined \textbf{explicitly} in terms of $x$ if $y$ appears only on the first side of the equation. For example, $y = 5x^2$ is an explicit function of $x$. However, sometimes, this may not be the case. A function $y$ is said to be \textbf{implicitly} defined as a function of $x$ if the earlier statement is not true. For example, $x^2+y^2=1$ implicitly defines $x$ as a function of $y$. 

Now, a more alert student may detect that the earlier example, $x^2+y^2=1$, is actually not a function of $x$, as it fails the vertical line test. This fact holds true across implicitly defined functions: equations that implicitly define functions may not represent functions themselves. They may implicitly define \textbf{more than one} function. For example, when solving for $y$ in the earlier example, we obtain $y=\pm \sqrt{1-x^2}$. Thus, the equation $x^2+y^2=1$ implicitly defines two functions: the upper semicircle and the lower semicircle.

To express it more rigorously, an equation of $x$ and $y$ defines the function $f(x)$ implicitly if the graph of $f(x)$ coincides with a part of the graph of the equation. 

\subsection{Utilization in Differentiation}
Implicit functions of $x$ can be differentiated similarly to how explicit functions can. It requires a bit more nuance in the handling of variables, though. Given an equation relating $x$ and $y$, the derivative of $y$ with respect to $x$ can be found without solving for $y$ explicitly. This is done by treating $y$ as a function of $x$ and taking the derivative of the whole equation using the chain rule. An example is shown below:

\[ xy = 1 \]
\[ x \cdot y(x) = 1 \]
\[ \frac{d}{dx} (x \cdot y(x)) = \frac{d}{dx}1 \]
\[ y(x) + x \cdot \frac{d}{dx}y(x) = 0 \]
\[ \frac{d}{dx}y(x) = -\frac{y(x)}{x} \]
\[ \frac{dy}{dx} = -\frac{y}{x} \]

Note that oftentimes, this will end up being our final answer. It is okay for our implicit derivative to contain $y$ in it. However, due to the simplicity of the original equation, $xy = 1$, we can explicitly solve for $y$ and plug it into the result above to obtain a derivative purely in terms of $x$.

\[ \frac{dy}{dx} = -\frac{\frac{1}{x}}{x} \]
\[ \frac{dy}{dx} = -\frac{1}{x^2} \]

We can verify this result by explicitly solving for $y$ and then taking the derivative:

\[ xy = 1 \]
\[ y = \frac{1}{x}\]
\[ \frac{d}{dx}y = \frac{d}{dx}(\frac{1}{x})\]
\[ \frac{dy}{dx} = -\frac{1}{x^2}\]

Implicit differentiation allows us to find derivatives from equations relating two variables. This ability has practical uses, such as in related rates problems, which will be explored below.

\section{Related Rates}
Related rates problems involve investigating the relationship between two quantities that are both changing over time or some other common variable. The format of and process of solving these problems requires the use of implicit differentiation and is easier displayed through an example than explained with words. 

\vspace{5}
Suppose the following:
\[  y=x^3, x=2 \enspace \textrm{and} \enspace \frac{dx}{dt}=4 \enspace \textrm{when} \enspace t=1\]
Find $\frac{dy}{dt}$ at time $t=1$. 
\vspace{5}

To solve a related rates problem, we first must utilize implicit differentiation to differentiate the relationship between the variables with respect to time to produce a relationship between their derivatives. 

\[  \frac{d}{dt}y(t) = \frac{d}{dt}(x(t))^3 \]
\[  \frac{dy}{dt} = 3x^2\frac{dx}{dt} \]

Next, we must utilize the rest of the information given in the problem to complete our solution. We know $x=2$ and $\frac{dx}{dt}=4$ when $t=1$, and we want to find $\frac{dy}{dt}$ when $t=1$. Therefore, we can simply plug in our known values and solve the equation:

\[ \textrm{at time} \enspace t=1, \frac{dy}{dt} = 3(2)^2(4) \]
\[ \frac{dy}{dt} = 48 \]

The process of solving related rates problems \textbf{rarely changes}: 

\begin{enumerate}
    \item You will be given a relationship and some information
    \item Implicitly differentiate the relationship 
    \item Plug in the information to solve the problem
\end{enumerate}

However, actual related rates problems will be contextual, and often will require the use of some problem solving intuition or previously learned formula (i.e. area or volume formulas or Pythagorean Theorem) to deduce the relationship between the variables of interest.

\end{document}