\documentclass[11pt]{article}
\usepackage[margin=1.5in]{geometry}
\usepackage{graphicx}
\usepackage{float}
\usepackage{parskip}
\usepackage{amsmath}

\usepackage{pgfplots}
\pgfplotsset{width=10cm, compat=1.9}
\usetikzlibrary{angles, quotes}


\begin{document}

\textbf{\Huge Newton's Laws of Motion}

Athan Zhang \& Jeffrey Chen

\section{Introduction}

Now that we understand basic motion, we can relate it to mass and force. Newton's laws of motion are fundamental principles in physics that describe the relationship between the motion of an object and the forces acting upon it. These laws were formulated by Sir Isaac Newton in the late 17th century and laid the foundation for classical mechanics, revolutionizing our understanding of how objects move and interact with each other. The laws can be summarized as follows
\begin{enumerate}
    \item \textbf{Newton's First Law of Motion (The Law of Inertia)}\newline An object at rest stays at rest unless acted on by an external force. An object in motion will continue moving in a straight line at a constant velocity unless acted on by an external force. This describes inertia, which is the property of an object to resist changes in its motion.
    \item \textbf{Newton's Second Law of Motion}\newline The acceleration of an object is directly proportional to the net force applied to it and inversely proportional to its mass. It is mathematically given as:
    \begin{align*}
        \Vec{F_{\text{net}}} = m\Vec{a}
    \end{align*}
    \item \textbf{Newton's Third Law of Motion (The Law of Action-Reaction)}\newline For every action, there is an equal and opposite reaction. When an object exerts a force on another object, the second object exerts an equal and opposite force back on the first object.
\end{enumerate}

We will cover more on each individual law in later sections.

\section{Newton's First Law: The Law of Inertia}

Newton's First Law essentially establishes the idea of inertia. Inertia is the idea that something will keep moving at a constant velocity forever unless external forces act upon it. This is often very hard to see in nature because there is almost always some force acting on an object, usually friction or gravity, but in a perfect vacuum, this law would hold true. 

An important concept to understand is \textbf{reference frames}. While we talked about reference frames when dealing with relative motion, it's also prevalent when dealing with forces. In physics, a reference frame is a coordinate system that is used to describe the position, motion, and interactions of objects. It provides a framework for measuring distances, time, and other physical quantities. There are two main types of reference frames: reference frames and inertial reference frames.
\begin{enumerate}
    \item \textbf{Reference Frame:}\newline
    A reference frame is any set of axes and a standard clock that defines the positions and motions of objects. It establishes a point of origin and a set of axes (usually Cartesian coordinates) that allow us to measure distances and directions relative to that point.
    \item \textbf{Inertial Reference Frame:}\newline
    An inertial reference frame is a particular type of reference frame that is not subject to acceleration or rotation. In an inertial reference frame, objects at rest remain at rest, and objects in motion continue moving at a constant velocity unless acted upon by external forces.
\end{enumerate}

Inertial reference frames keep Newton's First Law true without arbitrarily creating "forces" that are caused by an accelerating reference frame.

The First Law can be written as
\begin{align*}
    \text{If } F_{\text{net}} = 0 \text{ Then } a = 0 \text{ and } v \text{ is constant}
\end{align*}

\section{Newton's Second Law: Force and Mass}

Newton's Second Law is one of the more famous laws in physics because it can be beautifully expressed mathematically as
\begin{align*}
    \Vec{F_{\text{net}}} = m\Vec{a}
\end{align*}
A \textbf{force} is an external influence on an object that causes it to accelerate relative to an inertial reference frame. The \textbf{direction} of the force is the direction of the acceleration it causes if it is the only force acting on the object. The \textbf{magnitude} of the force is the product of the mass of the object and its subsequent acceleration. We commonly think of a force as a \textit{push} or \textit{pull}, similar to that exerted by our muscles.

The metric unit for force is the Newton (N)
\begin{align*}
    1 N = 1 \frac{kg \cdot m}{s^2}
\end{align*}

\textbf{Mass} is an intrinsic property of an object that measures its resistance to acceleration. That is, it is a measure of the object's inertia. It is also a measure of much matter composes an object. It's intrinsic in the sense that it does not depend on the location of the object, as the mass of an object will always be the same.

The metric unit for mass is the kilogram (kg).

\subsection{The Force Due to Gravity: Weight}

You may have commonly heard that weight and mass are not the same things, and that would be correct. Weight is the force of gravity acting upon mass. If we drop an object near the earth's surface from free fall with no air resistance, the only acceleration on the object would be due to gravity. This force is called weight $\Vec{w}$.
\begin{align*}
    \Vec{w} = m\Vec{g}
\end{align*}
Since $\Vec{g}$ is the same for all objects at any given point, the weight of an object must be proportional to its mass. 

However, since the \textbf{gravitational field} of the earth is not uniform around the earth, that is to say, $\Vec{g}$ is not the same at all points around the earth, we know that weight is not intrinsic like mass as it fluctuates based on location. Careful measurements have shown that the farther you are from the earth's surface, the weaker gravity is. This will be explained more in later sections.

\section{Newton's Third Law}

Newton's Third Law may be the most unintuitive, but as commonly described: "For every action, there is an equal and opposite reaction." 

When object A exerts a force on object B, object B exerts an equal and opposite force on object A. However, this is not to say the forces co-occur; the forces are acting on different objects. 

A typical example is a textbook on a table. The textbook exerts a weight force on the table, but of course, it doesn't simply fall through. The table exerts an equal and opposite force on the textbook, which causes it to be stationary on the table. Another example is how a runner moves forward. As a runner takes a step, they exert a force into the ground, which exerts a force back into the runner. This propels them forward, which is why we are able to move forward when running.

\section{Fundamental Forces}

In physics, four fundamental forces govern the interactions between particles and objects at the microscopic level. These forces are:

\begin{enumerate}
\item \textbf{Gravity}: The force of gravity is responsible for the attraction between objects with mass. It is a universal force that acts over long distances and is described by Newton's law of universal gravitation.

\item \textbf{Electromagnetism}: Electromagnetic force is responsible for the interactions between charged particles. It includes both electric and magnetic forces and is described by Maxwell's equations. This force governs phenomena such as the behavior of electrons in atoms and the interaction of magnets.

\item \textbf{Weak Nuclear Force}: The weak nuclear force involves processes such as radioactive decay and certain types of nuclear reactions. It is responsible for transforming elementary particles and is described by the theory of quantum electroweak interactions.

\item \textbf{Strong Nuclear Force}: The strong nuclear force, also known as the strong interaction, is the force that holds atomic nuclei together. It is responsible for the binding of protons and neutrons within the nucleus and is described by quantum chromodynamics (QCD).

\end{enumerate}

These four fundamental forces play a crucial role in understanding the behavior of matter and energy in the universe. Theoretical frameworks such as general relativity and quantum field theory have been developed to describe and unify these forces under a single comprehensive theory, but a complete unified theory of all forces, often referred to as the "Theory of Everything," is still an active area of research in modern physics.

All forces we observe at the macroscopic level are due either to the gravitational or electromagnetic force. For example, friction is caused by the slight pull between charged particles when two objects are in contact.

\section{Action-at-a-Distance and Contact Forces}

In physics, forces can be categorized as either \textit{action-at-a-distance} forces or \textit{contact forces}. These classifications describe how forces are transmitted between objects.

\subsection{Action-at-a-Distance Forces}

Action-at-a-distance forces are those that act on objects without the need for physical contact or a medium to transmit the force. The force is exerted over a distance and can affect objects that are not in direct contact with each other. The two most common examples of action-at-a-distance forces are:

\begin{enumerate}
\item \textbf{Gravity}: The force of gravity is an action-at-a-distance force. It attracts objects with mass towards each other without direct contact. For example, the gravitational force of the Earth pulls objects towards its center, giving them weight.

\item \textbf{Electromagnetic Force}: The electromagnetic force is another example of an action-at-a-distance force. It includes forces such as electrostatic forces (between charged objects) and magnetic forces (between magnets or moving charges). These forces can act on objects without physical contact or a medium between them.

\end{enumerate}

\subsection{Contact Forces}

Contact forces, as the name suggests, are forces that act on objects only when they are in direct physical contact. These forces require objects to be in proximity and in direct touch with each other. Some common examples of contact forces include:

\begin{enumerate}
\item \textbf{Normal Force}: The normal force is the force exerted by a surface to support the weight of an object resting on it. For example, when you place a book on a table, the table exerts an upward normal force to balance the downward force of gravity.

\item \textbf{Tension Force}: Tension force occurs when a string or rope pulls an object or any other flexible connector. It acts along the direction of the connector and transmits forces between objects in contact.

\item \textbf{Springs}: When a spring is compressed or extended by a small amount $\Delta x$, the force it exerts is found to be
\begin{align*}
    F_x = -k\Delta x
\end{align*}
This is known as Hooke's Law.

\end{enumerate}

Both action-at-a-distance forces and contact forces play significant roles in the interactions and dynamics of objects. Understanding these forces is crucial for studying mechanics, electromagnetism, and various other areas of physics.

\section{Free-Body Diagrams}

In physics, a free-body diagram is a visual representation used to analyze and understand the forces acting on an object. It helps to isolate the object of interest and identify the forces acting on it. Here's a step-by-step guide on how to set up and use free-body diagrams effectively:

\subsection*{Step 1: Identify the Object of Interest}

Begin by identifying the specific object or body for which you want to analyze the forces. It could be a single object or a specific part of a larger system.

\subsection*{Step 2: Isolate the Object}

Isolate the identified object from its surroundings by mentally removing any connections or attachments. This simplification allows you to focus solely on the forces acting on the object itself.

\subsection*{Step 3: Identify the Forces}

Identify and list all the forces that act on the isolated object. These forces can be categorized into two main types:

\subsection*{Step 4: Represent the Forces}

Represent each identified force as a vector originating from the object's center of mass. The length and direction of the vectors represent the magnitude and direction of the forces, respectively. Use clear labels to identify each force.

\subsection*{Step 5: Choose a Coordinate System}

Choose a suitable coordinate system to represent the directions of forces and motion. Typically, a Cartesian coordinate system (x-y axis) is used, with the x-axis aligned along the horizontal direction and the y-axis aligned along the vertical direction.

\subsection*{Step 6: Label the Axes}

Label the chosen coordinate axes on the free-body diagram. This helps establish a clear reference for describing the forces and their components in subsequent calculations.

\subsection*{Step 7: Analyze Equilibrium or Motion}

Based on the problem or situation, analyze whether the object is in equilibrium (static) or in motion (dynamic). For equilibrium, the sum of all forces acting on the object should be zero. For motion, consider the acceleration and net force acting on the object.

\subsection*{Step 8: Solve the Problem}

Use the free body diagram and the principles of Newton's laws of motion to solve the problem at hand. Break down forces into their components with respect to the axes, consider their effects on the object's motion, and apply appropriate mathematical equations as needed.

By following these steps, free-body diagrams provide a visual representation of the forces acting on an object, aiding in the analysis and understanding of the physics involved.

\end{document}
