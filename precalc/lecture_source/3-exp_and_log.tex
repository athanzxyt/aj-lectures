\documentclass[11pt]{article}
\usepackage[margin=1.5in]{geometry}
\usepackage{graphicx}
\usepackage{float}
\usepackage{parskip}
\usepackage{amsmath}
\usepackage{subfigure}
\usepackage{ulem}
\usepackage{pgfplots}
\pgfplotsset{width=10cm, compat=1.9}

\begin{document}

\textbf{\Huge Exponential and Logarithmic \newline Functions}

Athan Zhang \& Jeffrey Chen

\section{Exponential Functions}

Unlike power, radical, polynomial, and rational functions, which are examples of \textbf{algebraic functions} as they obtain values by adding, subtracting, multiplying, or dividing constants with the independent variable, exponential and logarithmic functions are \textbf{transcendental functions}. These are functions that cannot be expressed in terms of algebraic operations, thus \textit{transcending} past algebra. 

Exponential functions play a significant role in mathematics and the sciences. They are widely used to model phenomena such as population growth, radioactive decay, compound interest, and more. In this section, we will explore the properties and behavior of exponential functions.

\subsection{Definition}

An exponential function is a function of the form $f(x) = a \cdot b^x$, where $a$ and $b$ are constants and $b > 0$. The base $b$ is called the \textit{base of the exponential function}. The constant $a$ is the \textit{initial value} or the value of the function when $x = 0$.

\subsection{Graphs of Exponential Functions}

The graph of an exponential function depends on the values of $a$ and $b$. Here are some general properties:

\begin{itemize}
  \item If $0 < b < 1$, the graph will be decreasing and approach the $x$-axis as $x$ becomes larger.
  \item If $b > 1$, the graph will be increasing and become steeper as $x$ becomes larger.
\end{itemize}

\begin{center}
\begin{tikzpicture}
  \begin{axis}[
    xlabel=$x$,
    ylabel=$y$,
    xmin=-8,
    xmax=8,
    ymin=-8,
    ymax=8,
    domain=-8:8,
    samples=100,
    axis lines=middle, width=6cm, height=6cm,
    smooth]

    \addplot[blue, thick] {2^x} node[right] {};

  \end{axis}
\end{tikzpicture}
\hspace{3em}
\begin{tikzpicture}
  \begin{axis}[
    xlabel=$x$,
    ylabel=$y$,
    xmin=-8,
    xmax=8,
    ymin=-8,
    ymax=8,
    domain=-8:8,
    samples=100,
    axis lines=middle, width=6cm, height=6cm,
    smooth]

    \addplot[blue, thick] {0.5^x} node[right] {};

  \end{axis}
\end{tikzpicture}
\end{center}

\subsection{Key Properties}

Exponential functions have several important properties:

\begin{itemize}
  \item \textbf{Domain and Range:} The domain of any exponential function is $(-\infty, \infty)$, and the range depends on the sign of $a$. If $a > 0$, the range is $(0, \infty)$, while if $a < 0$, the range is $(-\infty, 0)$.
  \item \textbf{Continuity:} Any exponential function is continuous on $(-\infty, \infty)$.
  \item \textbf{End Behavior:} When $b > 1$, the function exhibits exponential growth with $\lim_{x \to -\infty} f(x) = 0$ and $\lim_{x \to \infty} f(x) = \infty$. When $0 < b < 1$, the function exhibits exponential decay with $\lim_{x \to -\infty} f(x) = \infty$ and $\lim_{x \to \infty} f(x) = 0$.
  \item \textbf{Asymptotes:} The graph of an exponential function never touches or crosses the $x$-axis (unless transformed). However, it may approach the $x$-axis as $x$ tends to $-\infty$ or $+\infty$, depending on the sign of $a$ and $b$.
\end{itemize}

\subsection{Natural Base}

Most real-world applications involving exponential functions use an irrational number $e$ called the \textbf{natural base} and is calculated as
\begin{align*}
    e = \lim_{x \to \infty}\left(1 + \frac{1}{x}\right)^{x}
\end{align*}
This value is approximated to be 
\begin{align*}
    e = 2.718281828\dots
\end{align*}
But we can simply use $e = 2.718$.

\subsection{Compound Interest}

A typical application of exponential growth is compound interest in finance. Suppose an initial principal $P$ is invested into an account with an annual interest rate of $r$, and the interest is compounded annually. This means that at the end of each year, the interest earned is added to the account balance, with the new sum becoming the principal for the following year.

\begin{table}[H]
    \centering
    \begin{tabular}{c|l}
        Year & Account Balance \\
        \hline
        0 & $A_{0} = P$ \\
        1 & $A_{1} = A_{0} + A_{0}r = P(1 + r)$ \\
        2 & $A_{2} = A_{1} + A_{1}r = P(1 + r)^2$ \\
        3 & $A_{2} = A_{2} + A_{2}r = P(1 + r)^3$ \\
        $t$ & $A_{t} = A_{t - 1} + A_{t - 1}r = P(1 + r)^t$
    \end{tabular}
\end{table}
As we can see, the pattern that develops leads to an exponential function with a base of $(1+r)$.

To allow for quarterly, monthly, or daily compounding, let $n$ be the number of times the interest is compounded each year. This means that the compounding rate is $\frac{r}{n}$, and there are $nt$ compoundings every year. We then obtain the general formula for compound interest
\begin{align*}
    A(t) = P\left(1 + \frac{r}{n}\right)^{nt}
\end{align*}
Where $P$ is the initial principal invested, $r$ is the annual interest rate, and $n$ is the number of times the interest is compounded in a year.

But what if interest was compounded \textit{continuously} such that $\lim_{n \to \infty}$? We can solve such a scenario by manipulating the general compound interest formula. 
\begin{align*}
    A(t) &= P\left(1 + \frac{r}{n}\right)^{nt} & \text{Rewrite $\frac{r}{n}$ as $\frac{1}{\frac{n}{r}}$} \\
         &= P\left(1 + \frac{1}{x}\right)^{xrt} & \text{Let $x = \frac{n}{r}$ and $n = xr$}\\  
         &= P\left[\left(1 + \frac{1}{x}\right)^{x}\right]^{rt} & \text{Power Property}
\end{align*}
The expression in brackets should look familiar as it is close to the definition of the natural base. Since $r$ is a fixed value and $x = \frac{n}{r}$, $x\to\infty$ as $n\to\infty$, the formula becomes
\begin{align*}
    \lim_{n \to \infty} P\left(1 + \frac{r}{n}\right)^{nt} = \lim_{x \to \infty} P\left[\left(1 + \frac{1}{x}\right)^{x}\right]^{rt} = Pe^{rt}
\end{align*}

\subsection{General Exponential Growth and Decay}

In addition to investments, populations of people, bacteria, and amounts of radioactive material can change at an exponential rate. The previously derived formulas for exponential growth and decay can be applied to any situation where growth is proportional to the initial size of the quantity being considered.

\begin{table}[H]
    \centering
    \begin{tabular}{|p{7cm}|p{7cm}|}
    \hline
        General Exponential Growth of Decay & Continuous Exponential Growth or Decay \\
        \hline
        $N = N_{0}(1 + r)^{t}$ & $N = N_{0}e^{kt}$ \\
        There is \textit{growth} if $r > 0$ and \textit{decay} if $r < 0$ & There is \textit{continuous growth} if $k > 0$ and \textit{continuous decay} if $k < 0$ \\
        \hline
    \end{tabular}
\end{table}

Where there is an initial quantity $N_{0}$ that grows or decays at an exponential rate of $r$ or $k$ with a final quantity of $N$ after $t$ units.

\section{Logarithmic Functions}

Logarithmic functions are the inverse of exponential functions and play a crucial role in various areas of mathematics and science. They are used to model phenomena such as exponential decay, data compression, signal processing, and more. 

\subsection{Definition}

A logarithmic function is a function of the form $f(x) = \log_b(x)$, where $b > 0$ and $b \neq 1$. The base $b$ determines the behavior of the logarithmic function. The logarithm function $\log_b(x)$ represents the exponent to which $b$ must be raised to obtain $x$.

\subsection{Graphs of Logarithmic Functions}

The graph of a logarithmic function depends on the base $b$. Here are some general properties:

\begin{itemize}
  \item If $0 < b < 1$, the graph will be decreasing and approach the $y$-axis as $x$ becomes larger.
  \item If $b > 1$, the graph will be increasing and become steeper as $x$ becomes larger.
\end{itemize}

\begin{center}
\begin{tikzpicture}
  \begin{axis}[
    xlabel=$x$,
    ylabel=$y$,
    xmin=-8,
    xmax=8,
    ymin=-8,
    ymax=8,
    domain=-8:8,
    samples=100,
    axis lines=middle, width=6cm, height=6cm,
    smooth]

    \addplot[blue, thick] {ln(x)} node[right] {};

  \end{axis}
\end{tikzpicture}
\hspace{3em}
\begin{tikzpicture}
  \begin{axis}[
    xlabel=$x$,
    ylabel=$y$,
    xmin=-8,
    xmax=8,
    ymin=-8,
    ymax=8,
    domain=-8:8,
    samples=100,
    axis lines=middle, width=6cm, height=6cm,
    smooth]

    \addplot[blue, thick] {ln(x)/ln(0.36)} node[right] {};

  \end{axis}
\end{tikzpicture}
\end{center}

\subsection{Key Properties}

Logarithmic functions have several important properties:

\begin{itemize}
  \item \textbf{Domain and Range:} The domain of any parent logarithmic function is $(0, \infty)$, while the range is $(-\infty, \infty)$.
  \item \textbf{Continuity:} Any parent logarithmic function is continuous on $(0, \infty)$.
  \item \textbf{End Behavior:} The end behavior of any parent logarithmic function is $\lim_{x \to 0^{+}} f(x) = -\infty$ and $\lim_{x \to \infty} f(x) = \infty$.
  \item \textbf{Asymptotes:} The graph of any logarithmic function never touches or crosses the $y$-axis (unless transformed). However, it may approach the $y$-axis as $y$ tends to $-\infty$ or $+\infty$, depending on the sign of $b$.
\end{itemize}

\subsection{Properties of Logarithms}

Logarithmic functions have several important properties that help simplify and manipulate logarithmic expressions. Here are some key properties:

\begin{itemize}
    \item \textbf{Product Rule: }$\log_{b}(M\cdot N) = \log_{b}(M) + \log_{b}(N)$
    \item \textbf{Quotient Rule: }$\log_{b}(\frac{M}{N}) = \log_{b}(M) - \log_{b}(N)$
    \item \textbf{Power Rule: }$\log_{b}(M^{k}) = k\cdot \log_{b}(M)$
    \item \textbf{Change of Base Rule: }$\log_{b}(M) = \frac{\log_{a}(M)}{\log_{a}(b)}$
    \item $\log_{b}(1) = 0$
    \item $\log_{b}(b) = 1$
    \item $\log_{b}(b^{k}) = k$
    \item $b^{\log_{b}(k)} = k$
\end{itemize}

\section{Exponential and Logarithmic Equations}
Exponential functions are one-to-one, meaning that no y-value is matched with more than one x-value. That is, $f(a) = f(b)$ \textbf{if and only if} $a = b$. This means we can solve simple exponential equations by expressing both sides of the equation in terms of a common base. 

\subsection*{Example:}
\begin{align*}
    36^{x+1} &= 6^{x+6} & \\
    (6^{2})^{x+1} &= 6^{x+6} & 6^{2} = 36\\
    6^{2x+2} &= 6^{x+6} & \text{Power of a Power}\\
    2x + 2 &= x+6 & \text{One-to-one Property}\\
    x + 2 &= 6 & \text{Subtract $x$ from both sides}\\
    x &= 4 & \text{Subtract 2 from both sides}\\
\end{align*}

The same principles can be applied to logarithmic equations since logarithmic equations are the inverses of exponential equations. One technique is to use exponentiation as follows
\begin{align*}
    \log_{2}x &= 3 & \text{Original Equation}\\
    2^{\log_{2}x} &= 2^{3} & \text{One-to-one Property}\\
    x &= 2^{3} & \text{Inverse Property}
\end{align*}
This converts the logarithmic equation into a much more manageable exponential form. This is not to say, however, that we can't solve logarithmic equations in a similar fashion to exponential equations. We can still use the one-to-one property to solve equations.

\subsection*{Example:}
\begin{align*}
    \log_{3}(x^2 + 16) &= 2\log_{3}5 & \\
    \log_{3}(x^2 + 16) &= \log_{3}25 & \text{Power Rule}\\
    x^2 + 16 &= 25 & \text{One-to-one Property}\\
    x^2 &= 9 & \text{Subtract 16 from both sides}\\
    x &= \pm 3 & \text{Take the square root of both sides}\\
\end{align*}

Remember that we have to check for extraneous solutions. Remember that the $x$ in $\log x$ cannot be negative as it does not exist. The following is an example of an equation that has an extraneous solution.

\subsection*{Example:}
\begin{align*}
    \log_{12} 12x + \log_{12}(x-1) &= 2 \\
    \log_{12} 12x(x-1) &= 2 & \text{Product Rule} \\
    \log_{12}(12x^2 - 12x) &= 2 & \text{Distributive Property} \\
    \log_{12}(12x^2 - 12x) &= \log_{12}12^2 & \text{Inverse Property} \\
    \log_{12}(12x^2 - 12x) &= \log_{12}144 & 12^2 = 144 \\
    12x^2 - 12x &= 144 & \text{One-to-one Property}\\
    12x^2 - 12x - 144 &= 0 & \text{Subtract 12 from both sides} \\
    12(x-4)(x+3) &= 0 & \text{Factor} \\
    x = 4 \text{ or } x &= -3
\end{align*}

When we check $x = 4$, we see that the solution is valid. However, when we check $x = -3$, we see that we get $\log_{12}(-36) + \log_{12}(-4) = 2$, which means the solution is extraneous as neither $\log_{12}(-36)$ or $\log_{12}(-4)$ are defined.

\end{document}
