\documentclass[tikz,border=10pt]{article}
\newcommand{\course}{Physics}
\usepackage[margin=1.5in]{geometry}
\newcommand{\instructor}{Athan Zhang \& Jeffrey Chen}
\usepackage{amsmath}
\usepackage{pgfplots}
\pgfplotsset{width=10cm, compat=1.9}
\usepackage[inline]{enumitem}
\pgfplotsset{compat=1.12}
\usepackage{float}
\usepackage{parskip}

% Document Body
\begin{document}

% Title
\begin{center}
\textbf{\Huge Problem Set 03} % Replace "#" with the problem set number
\end{center}

% Course and Instructor Information
\begin{flushleft}
\emph{Course:} \course \\
\emph{Instructors:} \instructor \\
\end{flushleft}

\section*{Problem 1}
An object experiences an acceleration of 3 m/s$^2$ when a certain force $F$ acts on it.
\begin{enumerate}
    \item What is the acceleration when the force is doubled?
    \item A second object experiences an acceleration of 9 m/s$^2$ under the influence of $F$, what is the ratio of the masses of the two objects. 
    \item If the two masses are tied together, what acceleration will the force $F$ produce?
\end{enumerate}

\section*{Problem 2}
How can you tell if a particular reference frame is an inertial reference frame? Assume you know the properties of the reference frame.

\section*{Problem 3}
If an object has no acceleration in an inertial reference frame, is it possible for forces to be acting on it?

\section*{Problem 4}
A body moves with constant speed in a straight line in an inertial reference frame. Which of the following statements must be true? (Circle all)
\begin{itemize}
    \item No force acts on the body.
    \item A single constant force acts on the body in the direction of motion.
    \item A single constant force acts on the body in the direction opposite to the motion.
    \item A net force of zero acts on the body.
    \item A constant net force acts on the body in the direction of motion.
\end{itemize}

\section*{Problem 5}
A force $\Vec{F} =$ ($6 \hat{i} - 3 \hat{j}$) N acts on an object of mass 2 kg. Find the magnitude and direction of $\Vec{a}$.

\section*{Problem 6}
A 4 kg object is subjected to two forces: $\Vec{F_1} =$ ($3 \hat{i} - 3 \hat{j}$) N and $\Vec{F_2} =$ ($-4 \hat{i} - 2 \hat{j}$) N. The object is at rest at the origin at time $t = 0$.
\begin{enumerate}
    \item What is the object's acceleration?
    \item What is the velocity at time $t = 3$s?
    \item Where is the object at time $t = 5$s?
\end{enumerate}

\section*{Problem 7}
Jeffrey finds himself on a mysterious foreign planet that seems like Earth. He is suspicious, however, so he drops a lead ball of mass 80g from the top of a cliff, 17m above the planet's surface. It takes 2.3s to reach the ground.
\begin{enumerate}
    \item What is Jeffrey's weight on this planet?
    \item Is he on Earth?
\end{enumerate}

\section*{Problem 8}
A 6 kg box on a frictionless horizontal surface is attached to a horizontal spring with a force constant of 800 N/m. If the spring is stretched 4 cm from its equilibrium length, what is the box's acceleration upon release?

\section*{Problem 9}
A particle of mass $m$ is traveling at an initial speed $v = 25.0$
m/s. It is brought to rest at a distance of 62.5m when a net force of 15.0 N acts on it. What is $m$?

\section*{Problem 10}
A rocket of mass 10kg starts from rest and exits from the 1.5m long barrel of a battleship turret at 1400 m/s. Find the force exerted on the rocket, assuming it to be constant, while the rocket is in the barrel.

\end{document}
